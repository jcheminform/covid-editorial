%% BioMed_Central_Tex_Template_v1.06
%%                                      %
%  bmc_article.tex            ver: 1.06 %
%                                       %

%%IMPORTANT: do not delete the first line of this template
%%It must be present to enable the BMC Submission system to
%%recognise this template!!

%%%%%%%%%%%%%%%%%%%%%%%%%%%%%%%%%%%%%%%%%
%%                                     %%
%%  LaTeX template for BioMed Central  %%
%%     journal article submissions     %%
%%                                     %%
%%          <8 June 2012>              %%
%%                                     %%
%%                                     %%
%%%%%%%%%%%%%%%%%%%%%%%%%%%%%%%%%%%%%%%%%


%%%%%%%%%%%%%%%%%%%%%%%%%%%%%%%%%%%%%%%%%%%%%%%%%%%%%%%%%%%%%%%%%%%%%
%%                                                                 %%
%% For instructions on how to fill out this Tex template           %%
%% document please refer to Readme.html and the instructions for   %%
%% authors page on the biomed central website                      %%
%% http://www.biomedcentral.com/info/authors/                      %%
%%                                                                 %%
%% Please do not use \input{...} to include other tex files.       %%
%% Submit your LaTeX manuscript as one .tex document.              %%
%%                                                                 %%
%% All additional figures and files should be attached             %%
%% separately and not embedded in the \TeX\ document itself.       %%
%%                                                                 %%
%% BioMed Central currently use the MikTex distribution of         %%
%% TeX for Windows) of TeX and LaTeX.  This is available from      %%
%% http://www.miktex.org                                           %%
%%                                                                 %%
%%%%%%%%%%%%%%%%%%%%%%%%%%%%%%%%%%%%%%%%%%%%%%%%%%%%%%%%%%%%%%%%%%%%%

%%% additional documentclass options:
%  [doublespacing]
%  [linenumbers]   - put the line numbers on margins

%%% loading packages, author definitions

%\documentclass[twocolumn]{bmcart}% uncomment this for twocolumn layout and comment line below
\documentclass{bmcart}

%%% Load packages
%\usepackage{amsthm,amsmath}
%\RequirePackage{natbib}
%\RequirePackage[authoryear]{natbib}% uncomment this for author-year bibliography
%\RequirePackage{hyperref}
\usepackage[utf8]{inputenc} %unicode support
%\usepackage[applemac]{inputenc} %applemac support if unicode package fails
%\usepackage[latin1]{inputenc} %UNIX support if unicode package fails

\usepackage{hyperref}
\usepackage{breakurl}
%%%%%%%%%%%%%%%%%%%%%%%%%%%%%%%%%%%%%%%%%%%%%%%%%
%%                                             %%
%%  If you wish to display your graphics for   %%
%%  your own use using includegraphic or       %%
%%  includegraphics, then comment out the      %%
%%  following two lines of code.               %%
%%  NB: These line *must* be included when     %%
%%  submitting to BMC.                         %%
%%  All figure files must be submitted as      %%
%%  separate graphics through the BMC          %%
%%  submission process, not included in the    %%
%%  submitted article.                         %%
%%                                             %%
%%%%%%%%%%%%%%%%%%%%%%%%%%%%%%%%%%%%%%%%%%%%%%%%%


\def\includegraphic{}
\def\includegraphics{}



%%% Put your definitions there:
\startlocaldefs
\endlocaldefs


%%% Begin ...
\begin{document}

%%% Start of article front matter
\begin{frontmatter}

%\begin{fmbox}
\dochead{Editorial}

%%%%%%%%%%%%%%%%%%%%%%%%%%%%%%%%%%%%%%%%%%%%%%
%%                                          %%
%% Enter the title of your article here     %%
%%                                          %%
%%%%%%%%%%%%%%%%%%%%%%%%%%%%%%%%%%%%%%%%%%%%%%

\title{What is the Role of Cheminformatics in a Pandemic?}

%%%%%%%%%%%%%%%%%%%%%%%%%%%%%%%%%%%%%%%%%%%%%%
%%                                          %%
%% Enter the authors here                   %%
%%                                          %%
%% Specify information, if available,       %%
%% in the form:                             %%
%%   <key>={<id1>,<id2>}                    %%
%%   <key>=                                 %%
%% Comment or delete the keys which are     %%
%% not used. Repeat \author command as much %%
%% as required.                             %%
%%                                          %%
%%%%%%%%%%%%%%%%%%%%%%%%%%%%%%%%%%%%%%%%%%%%%%

\author[
   addressref={aff1},                   % id's of addresses, e.g. {aff1,aff2}
   %corref={aff1},                       % id of corresponding address, if any
   %noteref={n1},                        % id's of article notes, if any
   email={rajarshi_guha@vrtx.com}   % email address
]{\inits{RG}\fnm{Rajarshi} \snm{Guha} 0000-0001-7403-8819}
\author[
   addressref={aff2},                   % id's of addresses, e.g. {aff1,aff2}
   %corref={aff1},                       % id of corresponding address, if any
   %noteref={n1},                        % id's of article notes, if any
   email={egon.willighagen@maastrichtuniversity.nl}   % email address
]{\inits{EW}\fnm{Egon} \snm{Willighagen} 0000-0001-7542-0286}


%%%%%%%%%%%%%%%%%%%%%%%%%%%%%%%%%%%%%%%%%%%%%%
%%                                          %%
%% Enter the authors' addresses here        %%
%%                                          %%
%% Repeat \address commands as much as      %%
%% required.                                %%
%%                                          %%
%%%%%%%%%%%%%%%%%%%%%%%%%%%%%%%%%%%%%%%%%%%%%%

\address[id=aff1]{%                           % unique id
  \orgname{Vertex Pharmaceuticals}, % university, etc
  \street{50 Northern Ave},                     %
  \postcode{02210}                                % post or zip code
  \city{Boston, MA},                              % city
  \cny{USA}                                    % country
}

\address[id=aff2]{%                           % unique id
  \orgname{Maastricht university}, % university, etc
  \street{Universiteitssingel 50},                     %
  % \postcode{}                                % post or zip code
  \city{6229 ER Maastricht},                              % city
  \cny{Netherlands}                                    % country
}

%%%%%%%%%%%%%%%%%%%%%%%%%%%%%%%%%%%%%%%%%%%%%%
%%                                          %%
%% Enter short notes here                   %%
%%                                          %%
%% Short notes will be after addresses      %%
%% on first page.                           %%
%%                                          %%
%%%%%%%%%%%%%%%%%%%%%%%%%%%%%%%%%%%%%%%%%%%%%%

\begin{artnotes}
%\note{Sample of title note}     % note to the article
%\note[id=n1]{Equal contributor} % note, connected to author
\end{artnotes}

%\end{fmbox}% comment this for two column layout

%%%%%%%%%%%%%%%%%%%%%%%%%%%%%%%%%%%%%%%%%%%%%%
%%                                          %%
%% The Abstract begins here                 %%
%%                                          %%
%% Please refer to the Instructions for     %%
%% authors on http://www.biomedcentral.com  %%
%% and include the section headings         %%
%% accordingly for your article type.       %%
%%                                          %%
%%%%%%%%%%%%%%%%%%%%%%%%%%%%%%%%%%%%%%%%%%%%%%

\begin{abstractbox}
\begin{abstract} 
\end{abstract}
\begin{keyword}
\end{keyword}
\end{abstractbox}

%
%\end{fmbox}% uncomment this for twcolumn layout

\end{frontmatter}

%%%%%%%%%%%%%%%%%%%%%%%%%%%%%%%%%%%%%%%%%%%%%%
%%                                          %%
%% The Main Body begins here                %%
%%                                          %%
%% Please refer to the instructions for     %%
%% authors on:                              %%
%% http://www.biomedcentral.com/info/authors%%
%% and include the section headings         %%
%% accordingly for your article type.       %%
%%                                          %%
%% See the Results and Discussion section   %%
%% for details on how to create sub-sections%%
%%                                          %%
%% use \cite{...} to cite references        %%
%%  \cite{koon} and                         %%
%%  \cite{oreg,khar,zvai,xjon,schn,pond}    %%
%%  \nocite{smith,marg,hunn,advi,koha,mouse}%%
%%                                          %%
%%%%%%%%%%%%%%%%%%%%%%%%%%%%%%%%%%%%%%%%%%%%%%

%%%%%%%%%%%%%%%%%%%%%%%%% start of article main body
% <put your article body there>

%%%%%%%%%%%%%%%%
%% Background %%
%%

The COVID-19 pandemic has led to a spike in research output
surrounding various aspects of the disease, ranging in scale from the
molecular to the population level.  There has been a
number of preprints in the field of cheminformatics that attempt to
address questions surrounding the disease. For example, numerous
virtual screening publications have proposed potentially interesting
candidates. Indeed, Open Source and Open Data has lowered the cost of
running cheminformatics experiments.

During a recent conversation with our Editorial Board we discussed the
possibility of a thematic issue in the Journal of Cheminformatics on
COVID-19. We decided not to pursue such an effort, mainly since much
of the work that might be topical for the journal was focused on
applying pre-existing pipelines to prioritize compounds as potential
candidates. As noted earlier, we have decided that other venues would
be more suitable to applications of such standard pipelines, In
addition, given the urgency of the situation we felt that
computational predictions without some form of experimental
verification would not be very useful.

\section*{How does cheminformatics contribute to virus research}

Even though we have decided to not focus on COVID-19 publications, we
believe that it is important to consider how cheminformatics can 
contribute to studies of pandemic diseases. We
highlight some features of the field that will enable researchers to
make use of chemical information in an efficient and rapid manner to
address future diseases.


Scientifically, there are numerous areas where cheminformatics plays a
role in anti-viral research, such as in virtual screening and
optimizing chemical structures. While we won't assume that a given
topical area will be more or less relevant for future research, we
believe that principles and processes (and their associated
infrastructure) within the cheminformaticscommunity can enable rapid
and robust research when the need arises.

\subsection*{Better prepared with more open cheminformatics}

While it is evident that Open Science and Open Data have enabled a
plethora of studies on various aspects of COVID-19, ensuring that
future efforts can build on current research will require broad
acceptance (and adherence) to community standards. This requires clear
articulation of the relevant standards (e.g., file formats) as well as
best practices. It is not much use if standards are defined, yet
researchers do not employ them. In our view, funders and journals can
play a critical role in ensuring that standards and best practices are
adhered to.

\subsection*{Annotated literature}

While not really a standard, the chemical literature can benefit from
annotation. The most obvious first step is to ensure that are
molecules referred to in a manuscript is associated with a SMILES or
InChI string. Ideally, structure files in some standard format would
also be included. While it may be onerous to as authors to annotate
non-chemical terms using formal vocabularies, this could be a service
provided by publishers. A combination of chemical identifiers plus
publisher-generated annotations would be a powerful way to enhance
reuse of published work.

\subsection*{Emergency Access}

Many publishers removed the paywalls to their COVID19-related
content. We stress this is not Open Science and at best
\emph{emergency Open Access}. Ccollected in the CORD-19 data set, the
papers are available for text and data mining [<cite>Q93445532</cite>]
which can still benefit population of open science resources
[REFS]. However, scholars may loose access to the literature after the
paywalls are put up again, making it hard to study the details behind
the knowledge they find in databases.


\subsection*{Interoperability}



\subsection*{Infrastructure}

Along with FAIR data, infrastructure that enables researchers to use
cheminformatics methods and toolkits, and then deploy solutions in a
straightforward fashion can enable reuse and thereby encourage
application and replication within the community. To a large degree,
Open Source toolkits have democratized access to cheminformatics
methods, and platforms such as KNIME enable easy deployment of
solutions. Resources such as Github and Bitbucket are also critical
infrastructure that enable the community to access, modify and
contribute to published code. We note that the infrastructure
described here should also apply to data. While version control
platforms may not be suitable for large datasets, resources such as
Figshare and Zenodo are capable to handling datasets and their
associated metadata.


\subsection*{New avenues for dissemination}

A final question that should be asked, is the journal article even the
right platform to share knowledge during a pandemic. While peer review
is important, it is not a guarantee of reliability. Indeed, several
COVID-19 articles have been retracted shortly after they were
published [REFS]. While it is unlikely that peer review will be
replaced, mechanisms to support curation and usage tracking will
enhance rapid publication processes.

Another aspect, relevant to rapid publication needs is that journal
articles suffer from the "version of record" limitation: once
published, making corrections is hard, though corrigenda and errata
are possible, they are rare. Efforts such as the Live Journal of
Computational Molecular Science (LiveCoMS) [REF], which support the
publication of "live" reviews (i.e., review articles that are updated
on an ongoing basis) highlight how published articles can undergo
continuous revision.


\section*{Conclusion}

The rapid appearance of computational research on COVID-19 attests
to the value of Open Source, Open Data and Open Science in
general. In this context, chemical information is a fundamental
component and by encouraging the use of standards and best practices,
cheminformatics plays a key role in the development of a foundation on
which future research on emerging diseases can be built upon.

\begin{backmatter}

\section*{Competing interests}
  The authors declare that they have no competing interests.


%%%%%%%%%%%%%%%%%%%%%%%%%%%%%%%%%%%%%%%%%%%%%%%%%%%%%%%%%%%%%
%%                  The Bibliography                       %%
%%                                                         %%
%%  Bmc_mathpys.bst  will be used to                       %%
%%  create a .BBL file for submission.                     %%
%%  After submission of the .TEX file,                     %%
%%  you will be prompted to submit your .BBL file.         %%
%%                                                         %%
%%                                                         %%
%%  Note that the displayed Bibliography will not          %%
%%  necessarily be rendered by Latex exactly as specified  %%
%%  in the online Instructions for Authors.                %%
%%                                                         %%
%%%%%%%%%%%%%%%%%%%%%%%%%%%%%%%%%%%%%%%%%%%%%%%%%%%%%%%%%%%%%

% if your bibliography is in bibtex format, use those commands:
\bibliographystyle{bmc-mathphys} % Style BST file (bmc-mathphys, vancouver, spbasic).
\bibliography{bmc_article}      % Bibliography file (usually '*.bib' )
% for author-year bibliography (bmc-mathphys or spbasic)
% a) write to bib file (bmc-mathphys only)
% @settings{label, options="nameyear"}
% b) uncomment next line
%\nocite{label}

% or include bibliography directly:
% \begin{thebibliography}
% \bibitem{b1}
% \end{thebibliography}

%%%%%%%%%%%%%%%%%%%%%%%%%%%%%%%%%%%
%%                               %%
%% Figures                       %%
%%                               %%
%% NB: this is for captions and  %%
%% Titles. All graphics must be  %%
%% submitted separately and NOT  %%
%% included in the Tex document  %%
%%                               %%
%%%%%%%%%%%%%%%%%%%%%%%%%%%%%%%%%%%





\end{backmatter}
\end{document}
