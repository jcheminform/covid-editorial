%% BioMed_Central_Tex_Template_v1.06
%%                                      %
%  bmc_article.tex            ver: 1.06 %
%                                       %

%%IMPORTANT: do not delete the first line of this template
%%It must be present to enable the BMC Submission system to
%%recognise this template!!

%%%%%%%%%%%%%%%%%%%%%%%%%%%%%%%%%%%%%%%%%
%%                                     %%
%%  LaTeX template for BioMed Central  %%
%%     journal article submissions     %%
%%                                     %%
%%          <8 June 2012>              %%
%%                                     %%
%%                                     %%
%%%%%%%%%%%%%%%%%%%%%%%%%%%%%%%%%%%%%%%%%


%%%%%%%%%%%%%%%%%%%%%%%%%%%%%%%%%%%%%%%%%%%%%%%%%%%%%%%%%%%%%%%%%%%%%
%%                                                                 %%
%% For instructions on how to fill out this Tex template           %%
%% document please refer to Readme.html and the instructions for   %%
%% authors page on the biomed central website                      %%
%% http://www.biomedcentral.com/info/authors/                      %%
%%                                                                 %%
%% Please do not use \input{...} to include other tex files.       %%
%% Submit your LaTeX manuscript as one .tex document.              %%
%%                                                                 %%
%% All additional figures and files should be attached             %%
%% separately and not embedded in the \TeX\ document itself.       %%
%%                                                                 %%
%% BioMed Central currently use the MikTex distribution of         %%
%% TeX for Windows) of TeX and LaTeX.  This is available from      %%
%% http://www.miktex.org                                           %%
%%                                                                 %%
%%%%%%%%%%%%%%%%%%%%%%%%%%%%%%%%%%%%%%%%%%%%%%%%%%%%%%%%%%%%%%%%%%%%%

%%% additional documentclass options:
%  [doublespacing]
%  [linenumbers]   - put the line numbers on margins

%%% loading packages, author definitions

%\documentclass[twocolumn]{bmcart}% uncomment this for twocolumn layout and comment line below
\documentclass{bmcart}

%%% Load packages
%\usepackage{amsthm,amsmath}
%\RequirePackage{natbib}
%\RequirePackage[authoryear]{natbib}% uncomment this for author-year bibliography
%\RequirePackage{hyperref}
\usepackage[utf8]{inputenc} %unicode support
%\usepackage[applemac]{inputenc} %applemac support if unicode package fails
%\usepackage[latin1]{inputenc} %UNIX support if unicode package fails

\usepackage{hyperref}
\usepackage{breakurl}


%%%%%%%%%%%%%%%%%%%%%%%%%%%%%%%%%%%%%%%%%%%%%%%%%
%%                                             %%
%%  If you wish to display your graphics for   %%
%%  your own use using includegraphic or       %%
%%  includegraphics, then comment out the      %%
%%  following two lines of code.               %%
%%  NB: These line *must* be included when     %%
%%  submitting to BMC.                         %%
%%  All figure files must be submitted as      %%
%%  separate graphics through the BMC          %%
%%  submission process, not included in the    %%
%%  submitted article.                         %%
%%                                             %%
%%%%%%%%%%%%%%%%%%%%%%%%%%%%%%%%%%%%%%%%%%%%%%%%%


\def\includegraphic{}
\def\includegraphics{}



%%% Put your definitions there:
\startlocaldefs
\endlocaldefs


%%% Begin ...
\begin{document}

%%% Start of article front matter
\begin{frontmatter}

%\begin{fmbox}
\dochead{Editorial}

%%%%%%%%%%%%%%%%%%%%%%%%%%%%%%%%%%%%%%%%%%%%%%
%%                                          %%
%% Enter the title of your article here     %%
%%                                          %%
%%%%%%%%%%%%%%%%%%%%%%%%%%%%%%%%%%%%%%%%%%%%%%

\title{What is the Role of Cheminformatics in a Pandemic?}

%%%%%%%%%%%%%%%%%%%%%%%%%%%%%%%%%%%%%%%%%%%%%%
%%                                          %%
%% Enter the authors here                   %%
%%                                          %%
%% Specify information, if available,       %%
%% in the form:                             %%
%%   <key>={<id1>,<id2>}                    %%
%%   <key>=                                 %%
%% Comment or delete the keys which are     %%
%% not used. Repeat \author command as much %%
%% as required.                             %%
%%                                          %%
%%%%%%%%%%%%%%%%%%%%%%%%%%%%%%%%%%%%%%%%%%%%%%

\author[
   addressref={aff1},                   % id's of addresses, e.g. {aff1,aff2}
   %corref={aff1},                       % id of corresponding address, if any
   %noteref={n1},                        % id's of article notes, if any
   email={rajarshi_guha@vrtx.com}   % email address
]{\inits{RG}\fnm{Rajarshi} \snm{Guha} 0000-0001-7403-8819}
\author[
   addressref={aff2},                   % id's of addresses, e.g. {aff1,aff2}
   %corref={aff1},                       % id of corresponding address, if any
   %noteref={n1},                        % id's of article notes, if any
   email={egon.willighagen@maastrichtuniversity.nl}   % email address
]{\inits{EW}\fnm{Egon} \snm{Willighagen} 0000-0001-7542-0286}


%%%%%%%%%%%%%%%%%%%%%%%%%%%%%%%%%%%%%%%%%%%%%%
%%                                          %%
%% Enter the authors' addresses here        %%
%%                                          %%
%% Repeat \address commands as much as      %%
%% required.                                %%
%%                                          %%
%%%%%%%%%%%%%%%%%%%%%%%%%%%%%%%%%%%%%%%%%%%%%%

\address[id=aff1]{%                           % unique id
  \orgname{Vertex Pharmaceuticals}, % university, etc
  \street{50 Northern Ave},                     %
  \postcode{02210}                                % post or zip code
  \city{Boston, MA},                              % city
  \cny{USA}                                    % country
}

\address[id=aff2]{%                           % unique id
  \orgname{Maastricht university}, % university, etc
  \street{Universiteitssingel 50},                     %
  % \postcode{}                                % post or zip code
  \city{6229 ER Maastricht},                              % city
  \cny{Netherlands}                                    % country
}

%%%%%%%%%%%%%%%%%%%%%%%%%%%%%%%%%%%%%%%%%%%%%%
%%                                          %%
%% Enter short notes here                   %%
%%                                          %%
%% Short notes will be after addresses      %%
%% on first page.                           %%
%%                                          %%
%%%%%%%%%%%%%%%%%%%%%%%%%%%%%%%%%%%%%%%%%%%%%%

\begin{artnotes}
%\note{Sample of title note}     % note to the article
%\note[id=n1]{Equal contributor} % note, connected to author
\end{artnotes}

%\end{fmbox}% comment this for two column layout

%%%%%%%%%%%%%%%%%%%%%%%%%%%%%%%%%%%%%%%%%%%%%%
%%                                          %%
%% The Abstract begins here                 %%
%%                                          %%
%% Please refer to the Instructions for     %%
%% authors on http://www.biomedcentral.com  %%
%% and include the section headings         %%
%% accordingly for your article type.       %%
%%                                          %%
%%%%%%%%%%%%%%%%%%%%%%%%%%%%%%%%%%%%%%%%%%%%%%

\begin{abstractbox}
\begin{abstract} 
\end{abstract}
\begin{keyword}
\end{keyword}
\end{abstractbox}

%
%\end{fmbox}% uncomment this for twcolumn layout

\end{frontmatter}

%%%%%%%%%%%%%%%%%%%%%%%%%%%%%%%%%%%%%%%%%%%%%%
%%                                          %%
%% The Main Body begins here                %%
%%                                          %%
%% Please refer to the instructions for     %%
%% authors on:                              %%
%% http://www.biomedcentral.com/info/authors%%
%% and include the section headings         %%
%% accordingly for your article type.       %%
%%                                          %%
%% See the Results and Discussion section   %%
%% for details on how to create sub-sections%%
%%                                          %%
%% use \cite{...} to cite references        %%
%%  \cite{koon} and                         %%
%%  \cite{oreg,khar,zvai,xjon,schn,pond}    %%
%%  \nocite{smith,marg,hunn,advi,koha,mouse}%%
%%                                          %%
%%%%%%%%%%%%%%%%%%%%%%%%%%%%%%%%%%%%%%%%%%%%%%

%%%%%%%%%%%%%%%%%%%%%%%%% start of article main body
% <put your article body there>


%% Points

% It might be an occasion to underscore the point that in vitro and ex
% vivo (cell-based) assays are themselves models, and that a signal virtue
% of checking predictions of different models is that each ones correct
% predictions are alike but that each one's errors are typically wrong in
% their own way.

% In my opinion, a chance to invoke the Anna Kerenina principle is a
% terrible thing to waste. :-)

% IMHO, most critical issues relate to what you touched on in the
% introductory paragraph, ie, expert use of cheminformatics approaches,
% strong linkage between computational predictions and experimental
% validation, and overall, rigor of cheminformatics-enabled testable
% hypothesis generation.

% I’d propose that addressing the issues of
% rigor, reproducibility, and practical impact of cheminformatics on
% pandemic research in an editorial with the title you proposed is much
% more to the point than the issues you highlighted.

% It would be useful
% for instance to offer rigorous guidance as to how computational papers
% should be evaluated when they are submitted to non-computational
% journals.

 

% Cheminformatics in action


% if the computational studies aren't good enough, why? What would be
% needed, what would this take, and how long (think eg the debates on
% how long the vaccine development should take). It's not clear to me,
% although I do cheminformatics, I do totally different things and would
% have no clue how to judge these drug discovery efforts

%  - tendency to heroism: everyone wants to bend their research to
%  covid, but is this always appropriate / wise / necessary? Why would
%  we need small molecule, untargeted mass spec when there's other, very
%  reliable tests out there


%%%%%%%%%%%%%%%%
%% Background %%
%%

The COVID-19 pandemic has led to a spike in research output
surrounding various aspects of the disease, ranging in scale from the
molecular to the population level.  There has been a number of
preprints in the field of cheminformatics that attempt to address
questions surrounding the disease. For example, numerous virtual
screening publications have proposed potentially interesting
candidates.

During a recent conversation with our Editorial Board we discussed the
possibility of a thematic issue in the Journal of Cheminformatics on
COVID-19. This editorial focuses on providing a more detailed
description behind this decision, as well as some thoughts on how
computational research should be evaluated when submitted to
non-computational venues.

\section*{Cheminformatics as a catalyst for anti-viral research}

The urgency of the COVID-19 epidemic presents a number of challenges
for the computational research community. While control and mitigaton of viral
spread is a primary focus of health systems, this is closely tied to
identifying pre-existing or novel therapeutic approaches to treat the
disease itself.

Cheminformatics approaches are one of the tools in the computational
toolbox that can be applied to therapeutic discovery. Given the
availability of Open Source and commercial tools, coupled with public
data, we have seen a large number of applications that have
prioritized compounds as potential therapeutic candidates. From the
point of view of this journal, straightforward applications of
pre-existing or well known pipelines are out scope for research
articles \cite{jcheminf_scope}.

However, one might argue that in such a crisis situation,
dissemination of all such applications could be beneficial. Indeed,
while more knowledge is useful in
the current pandemic, we believe that it needs to be rigorous
knowledge. A particularly egrigeous example are applications of drug
repurposing pipelines. Given the current state of the art it is very
easy to propose lists of approved or investigational drugs that could
serve as a COVID-19 therapeutic. While it is possible to make
justifications for some of these based on prior knowledge of mode of
action, it is still the fact that these are hypotheses. In our opinion,
the urgency of the current pandemic requires us to validate our
predictions with experimental results, and we can no longer carry on
(computational) business as usual.

While testable hypotheses are a key requirement in the current
setting, it is equally important that the pipeline used to reach such
hypotheses be as rigorous as possible. For statistical and machine
learning based approaches, appropriate statistical practice should be
employed \cite{cc_stats_1,cc_stats_2, cc_stats_3}. Similarly, best
practices should be employed for ligand-based \cite{qsar_1, qsar_2}
and structure-based approaches \cite{sbdd_1, sbdd_2}.

\subsection*{Cheminformatics in action}
\label{sec:chem-acti}

It is important to note that there are examples of work that
exemplifying collaboration between experimental and computational
groups tackling various aspects of COVID-19. We highlight two of them,
namely the COVID Moonshot \cite{moonshot} and the COVID-19 Molecular
Structure and Therapeutics Hub \cite{molssi}]. The COVID Moonshot
focuses on finding inhibitors of the Main protease of Covid19, and
involves 300 participants with a core group of 20 people. Importantly,
the project has access to computational, synthetic and bioassay
resources, coupled to a synchrotron that produces multiple structures
each week. The project has been able to identify starting points
exhibiting nanomolar potencies. While the project employs well known
computational methodologies, they key element is that computational
results are part of the design-make-test cycle. In other words,
computation is not isolated. Nonethless, the project makes use of key
Open Source products such as Fragalysis (a cloud-based application to
progress hits in fragment based drug design projects).


\subsection*{Open is necessary, but not sufficient}

While it is evident that Open Science and Open Data have enabled a
plethora of studies on various aspects of COVID-19, it is also true
that just making code and data available does not necessarily lead to
uptake. Why is this the case?  But it is important to remember that
even when data and methods are shared openly, they may not actually be
effective. This can be the case when a method is still in development
or has only been validated in a few specific systems. Similarly, data
may be preliminary and or generated in a very specific system. The
point is that in crisis situation, such as the COVID-19 pandemic,
researchers will tend to stick to what they know, but more
importantly, what they know works. One could argue that in such a
scenario, methodology development takes a back seat to data
publication, and ensuring that the relevant data is made available and
findable efficiently is a key task.


\section*{Conclusion}

It is clear that the computational chemistry and cheminformatics
community are actively engaged in COVID-19 research and the rapid
appearance of computational research on COVID-19 attests to the value
of Open Source, Open Data and Open Science in general. However, given
the scope of the journal, and the desire to encourage rigorous
cheminformatics studes, we have decided not to focus on COVID-19
related research submissions that are purely computational, with no
link to experimental validation. This does not preclude research that
addresses novel cheminformatics approaches and results, in the context
of COVID-19. We hope that cheminformatics researchers will consider
the role of chemical information, methods and standards in the context
of anti-viral research, and look forward to such submissions. But
given the urgency of the current situation, and the need to focus
resources on actionable outcomes, we have decided to not consider
applications of well known computational pipelines that simply result
in lists of putative inhibitors of a SARS-CoV-2 enzyme.



\begin{backmatter}

\section*{Competing interests}
  The authors declare that they have no competing interests.


%%%%%%%%%%%%%%%%%%%%%%%%%%%%%%%%%%%%%%%%%%%%%%%%%%%%%%%%%%%%%
%%                  The Bibliography                       %%
%%                                                         %%
%%  Bmc_mathpys.bst  will be used to                       %%
%%  create a .BBL file for submission.                     %%
%%  After submission of the .TEX file,                     %%
%%  you will be prompted to submit your .BBL file.         %%
%%                                                         %%
%%                                                         %%
%%  Note that the displayed Bibliography will not          %%
%%  necessarily be rendered by Latex exactly as specified  %%
%%  in the online Instructions for Authors.                %%
%%                                                         %%
%%%%%%%%%%%%%%%%%%%%%%%%%%%%%%%%%%%%%%%%%%%%%%%%%%%%%%%%%%%%%

% if your bibliography is in bibtex format, use those commands:
\bibliographystyle{bmc-mathphys} % Style BST file (bmc-mathphys, vancouver, spbasic).
\bibliography{bmc_article}      % Bibliography file (usually '*.bib' )
% for author-year bibliography (bmc-mathphys or spbasic)
% a) write to bib file (bmc-mathphys only)
% @settings{label, options="nameyear"}
% b) uncomment next line
%\nocite{label}

% or include bibliography directly:
% \begin{thebibliography}
% \bibitem{b1}
% \end{thebibliography}

%%%%%%%%%%%%%%%%%%%%%%%%%%%%%%%%%%%
%%                               %%
%% Figures                       %%
%%                               %%
%% NB: this is for captions and  %%
%% Titles. All graphics must be  %%
%% submitted separately and NOT  %%
%% included in the Tex document  %%
%%                               %%
%%%%%%%%%%%%%%%%%%%%%%%%%%%%%%%%%%%





\end{backmatter}
\end{document}
