%% BioMed_Central_Tex_Template_v1.06
%%                                      %
%  bmc_article.tex            ver: 1.06 %
%                                       %

%%IMPORTANT: do not delete the first line of this template
%%It must be present to enable the BMC Submission system to
%%recognise this template!!

%%%%%%%%%%%%%%%%%%%%%%%%%%%%%%%%%%%%%%%%%
%%                                     %%
%%  LaTeX template for BioMed Central  %%
%%     journal article submissions     %%
%%                                     %%
%%          <8 June 2012>              %%
%%                                     %%
%%                                     %%
%%%%%%%%%%%%%%%%%%%%%%%%%%%%%%%%%%%%%%%%%


%%%%%%%%%%%%%%%%%%%%%%%%%%%%%%%%%%%%%%%%%%%%%%%%%%%%%%%%%%%%%%%%%%%%%
%%                                                                 %%
%% For instructions on how to fill out this Tex template           %%
%% document please refer to Readme.html and the instructions for   %%
%% authors page on the biomed central website                      %%
%% http://www.biomedcentral.com/info/authors/                      %%
%%                                                                 %%
%% Please do not use \input{...} to include other tex files.       %%
%% Submit your LaTeX manuscript as one .tex document.              %%
%%                                                                 %%
%% All additional figures and files should be attached             %%
%% separately and not embedded in the \TeX\ document itself.       %%
%%                                                                 %%
%% BioMed Central currently use the MikTex distribution of         %%
%% TeX for Windows) of TeX and LaTeX.  This is available from      %%
%% http://www.miktex.org                                           %%
%%                                                                 %%
%%%%%%%%%%%%%%%%%%%%%%%%%%%%%%%%%%%%%%%%%%%%%%%%%%%%%%%%%%%%%%%%%%%%%

%%% additional documentclass options:
%  [doublespacing]
%  [linenumbers]   - put the line numbers on margins

%%% loading packages, author definitions

%\documentclass[twocolumn]{bmcart}% uncomment this for twocolumn layout and comment line below
\documentclass{bmcart}

%%% Load packages
%\usepackage{amsthm,amsmath}
%\RequirePackage{natbib}
%\RequirePackage[authoryear]{natbib}% uncomment this for author-year bibliography
%\RequirePackage{hyperref}
\usepackage[utf8]{inputenc} %unicode support
%\usepackage[applemac]{inputenc} %applemac support if unicode package fails
%\usepackage[latin1]{inputenc} %UNIX support if unicode package fails

\usepackage{hyperref}
\usepackage{breakurl}

\usepackage{lineno}
\linenumbers


%%%%%%%%%%%%%%%%%%%%%%%%%%%%%%%%%%%%%%%%%%%%%%%%%
%%                                             %%
%%  If you wish to display your graphics for   %%
%%  your own use using includegraphic or       %%
%%  includegraphics, then comment out the      %%
%%  following two lines of code.               %%
%%  NB: These line *must* be included when     %%
%%  submitting to BMC.                         %%
%%  All figure files must be submitted as      %%
%%  separate graphics through the BMC          %%
%%  submission process, not included in the    %%
%%  submitted article.                         %%
%%                                             %%
%%%%%%%%%%%%%%%%%%%%%%%%%%%%%%%%%%%%%%%%%%%%%%%%%


\def\includegraphic{}
\def\includegraphics{}



%%% Put your definitions there:
\startlocaldefs
\endlocaldefs


%%% Begin ...
\begin{document}

%%% Start of article front matter
\begin{frontmatter}

%\begin{fmbox}
\dochead{Editorial}

%%%%%%%%%%%%%%%%%%%%%%%%%%%%%%%%%%%%%%%%%%%%%%
%%                                          %%
%% Enter the title of your article here     %%
%%                                          %%
%%%%%%%%%%%%%%%%%%%%%%%%%%%%%%%%%%%%%%%%%%%%%%

\title{What is the Role of Cheminformatics in a Pandemic?}

%%%%%%%%%%%%%%%%%%%%%%%%%%%%%%%%%%%%%%%%%%%%%%
%%                                          %%
%% Enter the authors here                   %%
%%                                          %%
%% Specify information, if available,       %%
%% in the form:                             %%
%%   <key>={<id1>,<id2>}                    %%
%%   <key>=                                 %%
%% Comment or delete the keys which are     %%
%% not used. Repeat \author command as much %%
%% as required.                             %%
%%                                          %%
%%%%%%%%%%%%%%%%%%%%%%%%%%%%%%%%%%%%%%%%%%%%%%

\author[
   addressref={aff1},                   % id's of addresses, e.g. {aff1,aff2}
   %corref={aff1},                       % id of corresponding address, if any
   %noteref={n1},                        % id's of article notes, if any
   email={rajarshi_guha@vrtx.com}   % email address
]{\inits{RG}\fnm{Rajarshi} \snm{Guha} 0000-0001-7403-8819}
\author[
   addressref={aff2},                   % id's of addresses, e.g. {aff1,aff2}
   %corref={aff1},                       % id of corresponding address, if any
   %noteref={n1},                        % id's of article notes, if any
   email={egon.willighagen@maastrichtuniversity.nl}   % email address
]{\inits{EW}\fnm{Egon} \snm{Willighagen} 0000-0001-7542-0286}
\author[
   addressref={aff3},                   % id's of addresses, e.g. {aff1,aff2}
   %corref={aff1},                       % id of corresponding address, if any
   %noteref={n1},                        % id's of article notes, if any
   email={barbara.zdrazil@univie.ac.at}   % email address
]{\inits{BZ}\fnm{Barbara} \snm{Zdrazil} 0000-0001-9395-1515}
\author[
   addressref={aff4},                   % id's of addresses, e.g. {aff1,aff2}
   %corref={aff1},                       % id of corresponding address, if any
   %noteref={n1},                        % id's of article notes, if any
   email={jeliazkova.nina@gmail.com}   % email address
]{\inits{NJ}\fnm{Nina} \snm{Jeliazkova} 0000-0002-4322-6179}


\address[id=aff1]{%                           % unique id
  \orgname{Vertex Pharmaceuticals}, % university, etc
  \street{50 Northern Ave},                     %
  \postcode{02210}                                % post or zip code
  \city{Boston, MA},                              % city
  \cny{USA}                                    % country
}

\address[id=aff2]{%                           % unique id
  \orgname{Maastricht University}, % university, etc
  \street{Universiteitssingel 50},                     %
  % \postcode{}                                % post or zip code
  \city{6229 ER Maastricht},                              % city
  \cny{Netherlands}                                    % country
}

\address[id=aff3]{%                           % unique id
  \orgname{University of Vienna}, % university, etc
  \street{Althanstraße 14},                     %
  % \postcode{}                                % post or zip code
  \city{1090 Vienna},                              % city
  \cny{Austria}                                    % country
}

\address[id=aff4]{%                           % unique id
  \orgname{Ideaconsult Ltd,}, % university, etc
 %  \street{Althanstraße 14},                     %
  % \postcode{}                                % post or zip code
  \city{Sofia, 1000}, 
  \cny{Bulgaria}                                    % country
}



\begin{artnotes}
%\note{Sample of title note}     % note to the article
%\note[id=n1]{Equal contributor} % note, connected to author
\end{artnotes}

%\end{fmbox}% comment this for two column layout

\begin{abstractbox}
  \begin{abstract}
    The COVID-19 pandemic has led to a spike in research output
    \cite{covidlit} surrounding various aspects of the disease,
    ranging in scale from the molecular to the population level.
    There have been a number of preprints (and subsequent journal
    publications) in the field of cheminformatics that attempt to
    address questions surrounding the disease. For example, numerous
    virtual screening publications have proposed potentially
    interesting candidates (for an overview see this Scholia page
    \cite{scholia}). During a recent conversation with our Editorial
    Board we discussed the possibility of a thematic issue in the
    Journal of Cheminformatics on COVID-19. We concluded that we would
    not pursue such an issue and this editorial focuses on providing a
    more detailed description behind this decision.

\end{abstract}
\begin{keyword}
  cheminformatics, COVID-19, virtual screening, pandemic, SARS-CoV-2
\end{keyword}
\end{abstractbox}

%
%\end{fmbox}% uncomment this for twcolumn layout

\end{frontmatter}



\section*{Computing results is necessary, but not sufficient}

The urgency of the COVID-19 epidemic presents a number of challenges
for the computational research community. While control and mitigation
of viral spread is a primary focus of health systems, this is closely
tied to identifying pre-existing or novel therapeutic approaches to
treat the disease itself. 

Cheminformatics approaches are one set of tools in the computational
toolbox that can be applied to therapeutic discovery. Given the
availability of Open Source and commercial tools, coupled with public
data, we have seen many studies that have prioritized compounds as
potential therapeutic candidates. From the point of view of this
journal, straightforward applications of pre-existing or well-known
pipelines are out of scope for research articles \cite{jcheminf_scope}.

However, one might argue that in such a crisis situation,
dissemination of all such applications could be beneficial. Indeed,
while more knowledge is useful in the current pandemic, we believe
that it needs to be rigorous knowledge. A particularly egrigeous
example is applications of drug repurposing pipelines. Given the
current state of the art it is very easy to propose lists of approved
or investigational drugs, that could serve as COVID-19
therapeutics. While it is possible to make justifications for some of
these based on prior knowledge of mode of action, it still remains
that these are \emph{hypotheses}. In our opinion, the urgency of the
current pandemic requires us to validate our predictions with
experimental results, and we can no longer carry on (computational)
business as usual.

While testable hypotheses are a key requirement in the current
setting, it is equally important that the pipeline used to reach such
hypotheses be as rigorous as possible. For statistical and machine
learning based approaches, appropriate statistical methodology should
be employed \cite{cc_stats_2,cc_stats_1, cc_stats_3}. Similarly, best
practices should be followed for ligand-based \cite{qsar_1, qsar_2}
and structure-based approaches \cite{sbdd_1, sbdd_2}. While we expect
all work submitted to the journal to adhere to these practices, these aspects
become even more important when computational work with experimental
components are submitted to non-computational journals.


\section*{Open is a starting point}

The challenge of the current pandemic is to identify novel
therapeutics in a rapid but rigorous fashion. This suggests that novel
method development may not be well suited to the current scenario. On
the other hand, data being generated, either experimentally or
computationally can serve as a foundation for computational
studies. This is enabled by ensuring such data is made available in an
open fashion and following FAIR principles \cite{fair-1, fair-2}. For
example, see the COVID-19 Wikiproject \cite{wikidata} that aims to
make drug discovery related data FAIR. Yet, it is important to
remember that even when data and methods are shared openly, they may
not actually be effective, as in a crisis situation, such as the
COVID-19 pandemic, researchers will tend to stick to what they know.
But more importantly, they will tend to stick to what they know
works. We believe that in such a scenario, methodology development
takes a back seat to data publication, and ensuring that the relevant
data is made available and findable efficiently is a key task.

\section*{Cheminformatics in action}
\label{sec:chem-acti}

It is important to note that there are examples of work that exemplify
computational-experimental collaboration and the community response to
the plethora of computational studies and data. We highlight two of
them, namely the COVID Moonshot \cite{moonshot} and the COVID-19
Molecular Structure and Therapeutics Hub \cite{molssi}. The COVID
Moonshot focuses on finding inhibitors of the Main protease (Mpro) of
COVID-19, and involves 300 participants with a core group of 20
people. Importantly, the project has access to computational,
synthetic and bioassay resources, coupled to a synchrotron that
produces multiple structures each week. The project has been able to
identify Mpro binders exhibiting nanomolar potencies. While the
project employs well-known computational methodologies, the key
element is that computational results are part of the design-make-test
cycle. In other words, computation is not isolated. Nonetheless, the
project makes use of key Open Source products such as Fragalysis (a
cloud-based application to progress hits in fragment-based drug design
projects) \cite{fragalysis}. The Hub, on the other hand, is an example
of a resource that aggregates data of various types that can be used
for computational studies. This includes structure models, simulation
related datasets (e.g., configuration files, trajectories). While this
does not directly lead to tested small molecules, it represents an
invaluable coordinated resource covering the wide variety of
computational studies that being published.


\section*{Conclusion}

It is clear that the computational chemistry and cheminformatics
community are actively engaged in COVID-19 research and the rapid
appearance of computational research \cite{jcim_covid_editorial} on
COVID-19 attests to the value of Open Data, Open Source and Open
Science in general. However, given the scope of this journal, and the
desire to encourage rigorous cheminformatics studies, we have decided
not to created a thematic issue focused on COVID-19 related research
submissions that are purely computational screens. This does not
preclude research that addresses novel cheminformatics approaches and
results, in the context of COVID-19.

Indeed, the above discussion might suggest that the \emph{only} role
of cheminformatics is to identify new therapeutic interventions. While
this is a key and pressing role in the current pandemic, there are
many other areas such as databases (e.g., ChEMBL \cite{chembl}, which
underlies a number of virtual screening studies), protein features
such as post-translational modifications, force fields,
physicochemical properties and associated models that underly many of
the approaches that one can use to identify new therapeutics. We would
argue that these foundational areas are also critical to ensuring that
when the need arises for computational methods to be applied to
therapeutic development, that can do so on a solid foundation.

Thus, we hope that cheminformatics researchers will consider the role
of chemical information, methods and standards in the context of
anti-viral research, and look forward to such submissions. But given
the urgency of the current situation, and the need to focus resources
on actionable outcomes, we have decided to not consider applications
of well-known computational pipelines that simply result in lists of
putative inhibitors of a SARS-CoV-2 enzyme without experimental
validation.



\begin{backmatter}

\section*{Competing interests}
  The authors declare that they have no competing interests.


%%%%%%%%%%%%%%%%%%%%%%%%%%%%%%%%%%%%%%%%%%%%%%%%%%%%%%%%%%%%%
%%                  The Bibliography                       %%
%%                                                         %%
%%  Bmc_mathpys.bst  will be used to                       %%
%%  create a .BBL file for submission.                     %%
%%  After submission of the .TEX file,                     %%
%%  you will be prompted to submit your .BBL file.         %%
%%                                                         %%
%%                                                         %%
%%  Note that the displayed Bibliography will not          %%
%%  necessarily be rendered by Latex exactly as specified  %%
%%  in the online Instructions for Authors.                %%
%%                                                         %%
%%%%%%%%%%%%%%%%%%%%%%%%%%%%%%%%%%%%%%%%%%%%%%%%%%%%%%%%%%%%%

% if your bibliography is in bibtex format, use those commands:
\bibliographystyle{bmc-mathphys} % Style BST file (bmc-mathphys, vancouver, spbasic).
\bibliography{bmc_article}      % Bibliography file (usually '*.bib' )
% for author-year bibliography (bmc-mathphys or spbasic)
% a) write to bib file (bmc-mathphys only)
% @settings{label, options="nameyear"}
% b) uncomment next line
%\nocite{label}

% or include bibliography directly:
% \begin{thebibliography}
% \bibitem{b1}
% \end{thebibliography}

%%%%%%%%%%%%%%%%%%%%%%%%%%%%%%%%%%%
%%                               %%
%% Figures                       %%
%%                               %%
%% NB: this is for captions and  %%
%% Titles. All graphics must be  %%
%% submitted separately and NOT  %%
%% included in the Tex document  %%
%%                               %%
%%%%%%%%%%%%%%%%%%%%%%%%%%%%%%%%%%%





\end{backmatter}
\end{document}
